%%%%%%%%%%%%%%%%%%%%%%%%%%%%%%%%%%%%%%%%%%%%%%%%%%%%%%%%%%%%%%%%%%%%%%%%%%
\label{'pronunciation_alphabet'}
\section{Alphabet, Punctuation Marks}
%%%%%%%%%%%%%%%%%%%%%%%%%%%%%%%%%%%%%%%%%%%%%%%%%%%%%%%%%%%%%%%%%%%%%%%%%%
%
% There are \textbf{fourteen letters} in the Toki Pona alphabet.
%%%%%%%%%%%%%%%%%%%%%%%%%%%%%%%%%%%%%%%%%%%%%%%%%%%%%%%%%%%%%%%%%%%%%%%%%%
\index{Alphabet}
\index{Pronunciation}
\index{Consonants}
\subsection*{Consonants}
%%%%%%%%%%%%%%%%%%%%%%%%%%%%%%%%%%%%%%%%%%%%%%%%%%%%%%%%%%%%%%%%%%%%%%%%%%
%
Except for \textit{j}, all the consonants are pronounced like in English. 
\textit{j} is always pronounced just like the letter \textit{y}. 

\begin{supertabular}{p{2cm}|ll}
\textbf{letter}   &&    \textbf{pronounced as in} \\ % no-dictionary
k && \textbf{k}ill  \\ % no-dictionary
l && \textbf{l}et   \\ % no-dictionary
m && \textbf{m}et   \\ % no-dictionary
n && \textbf{n}et   \\ % no-dictionary
p && \textbf{p}it   \\ % no-dictionary
s && \textbf{s}ink  \\ % no-dictionary
t && \textbf{t}oo   \\ % no-dictionary
w && \textbf{w}et   \\ % no-dictionary
j && \textbf{y}et   \\ % no-dictionary
\end{supertabular} 

%%%%%%%%%%%%%%%%%%%%%%%%%%%%%%%%%%%%%%%%%%%%%%%%%%%%%%%%%%%%%%%%%%%%%%%%%%
\index{Vowels}
\subsection*{Vowels}
%%%%%%%%%%%%%%%%%%%%%%%%%%%%%%%%%%%%%%%%%%%%%%%%%%%%%%%%%%%%%%%%%%%%%%%%%%
%
Toki Pona's vowels are quite unlike English's. Whereas vowels in English are quite arbitrary and can be pronounced many different ways depending on the word, Toki Pona's vowels are all regular and never change pronunciation. 
If you're familiar with Italian, Spanish, Esperanto, or certain other languages, then your work is already done. The vowels are the same in Toki Pona as they are in these languages. 

\begin{supertabular}{p{2cm}|ll}
\textbf{letter}   &&    \textbf{pronounced as in} \\ % no-dictionary
a   &&    f\textbf{a}ther \\ % no-dictionary
e   &&    m\textbf{e}t    \\ % no-dictionary
i   &&    p\textbf{ee}l   \\ % no-dictionary
o   &&    m\textbf{o}re   \\ % no-dictionary
u   &&    f\textbf{oo}d   \\ % no-dictionary
\end{supertabular} 

%
%%%%%%%%%%%%%%%%%%%%%%%%%%%%%%%%%%%%%%%%%%%%%%%%%%%%%%%%%%%%%%%%%%%%%%%%%%
\subsection*{The More Advanced Stuff}
%%%%%%%%%%%%%%%%%%%%%%%%%%%%%%%%%%%%%%%%%%%%%%%%%%%%%%%%%%%%%%%%%%%%%%%%%%
%
All official Toki Pona words are never capitalized. 
They are lowercase even at the beginning of the sentence! 
The only time that capital letters are used is when you are using unofficial words, like the names of people or places or religions. 
%
%
%%%%%%%%%%%%%%%%%%%%%%%%%%%%%%%%%%%%%%%%%%%%%%%%%%%%%%%%%%%%%%%%%%%%%%%%%%
\subsection*{Special Characters}
%%%%%%%%%%%%%%%%%%%%%%%%%%%%%%%%%%%%%%%%%%%%%%%%%%%%%%%%%%%%%%%%%%%%%%%%%%
%

\begin{supertabular}{p{2cm}|ll}
\textbf{.} && \textit{separator}: A declarative sentence ends with a full stop. \\ % no-dictionary
\textbf{!} && \textit{separator}: An imperative or an interjection sentence ends with \\ &&  an exclamation mark. \\ % no-dictionary
\textbf{?} && \textit{separator}: An questions always ends in a question mark. \\ % no-dictionary
\textbf{:} && \textit{separator}: A colon is between an hint sentences and a sentences. \\  % no-dictionary
\textbf{,} && \textit{separator}: A comma is used after an 'o' to address people. \\ % no-dictionary
\end{supertabular} \\

%
%%%%%%%%%%%%%%%%%%%%%%%%%%%%%%%%%%%%%%%%%%%%%%%%%%%%%%%%%%%%%%%%%%%%%%%%%%
\subsection*{Separators}
%%%%%%%%%%%%%%%%%%%%%%%%%%%%%%%%%%%%%%%%%%%%%%%%%%%%%%%%%%%%%%%%%%%%%%%%%%
%
In these lessons, special characters are referred to as separators. 
Separators separate phrases from each other. 
For example, a dot separates a sentence from the next sentence. 
In Toki Pona there are also special words which serve as separators. 

%
%%%%%%%%%%%%%%%%%%%%%%%%%%%%%%%%%%%%%%%%%%%%%%%%%%%%%%%%%%%%%%%%%%%%%%%%%%
\subsection*{Types of Sentences}
%%%%%%%%%%%%%%%%%%%%%%%%%%%%%%%%%%%%%%%%%%%%%%%%%%%%%%%%%%%%%%%%%%%%%%%%%%
%
Like many languages, Toki Pona has different types of sentences. 

Most sentences are declarative sentences and end with a period. 
Declarative sentences make statements or assumptions. 

Question sentences are interrogative sentences that formulate a question. 
They end with a question mark. 

Imperative sentences are sentences that formulate a command. 
They end with an exclamation mark.

Exclamatory sentences (interjections) are sentences that express admiration or astonishment. 
This also includes greetings. 
They end with an exclamation mark or a period.

Headlines (titles) are usually not complete sentences and do not end with a punctuation mark.

Please always pay attention to correct punctuation marks. Wrong or missing
Punctuation marks impair the intelligibility.

%
\newpage
%%%%%%%%%%%%%%%%%%%%%%%%%%%%%%%%%%%%%%%%%%%%%%%%%%%%%%%%%%%%%%%%%%%%%%%%%%
\subsection*{Practice (Answers: Page~\pageref{'pronunciation_alphabet'})}
%%%%%%%%%%%%%%%%%%%%%%%%%%%%%%%%%%%%%%%%%%%%%%%%%%%%%%%%%%%%%%%%%%%%%%%%%%

Please write down your answers and check them afterwards. 

\begin{supertabular}{p{5cm}|ll}
What are separators? &&                                      \\ % no-dictionary
 && \\ % no-dictionary
Which phrase has no punctuation character at the end? &&     \\ % no-dictionary
 && \\ % no-dictionary
Which separator is at the end of a declarative sentence? &&  \\ % no-dictionary
 && \\ % no-dictionary
When are official \textit{toki pona} words capitalized? &&   \\ % no-dictionary
 && \\ % no-dictionary
What is usually not allowed before or after a separator? &&  \\ % no-dictionary

\end{supertabular} \\% no-dictionary
%
%%%%%%%%%%%%%%%%%%%%%%%%%%%%%%%%%%%%%%%%%%%%%%%%%%%%%%%%%%%%%%%%%%%%%%%%%%
% eof

%%%%%%%%%%%%%%%%%%%%%%%%%%%%%%%%%%%%%%%%%%%%%%%%%%%%%%%%%%%%%%%%%%%%%%%%%%
\section{Numbers}
%
\index{Number}
%
%%%%%%%%%%%%%%%%%%%%%%%%%%%%%%%%%%%%%%%%%%%%%%%%%%%%%%%%%%%%%%%%%%%%%%%%%%
\subsection*{Vocabulary}
%%%%%%%%%%%%%%%%%%%%%%%%%%%%%%%%%%%%%%%%%%%%%%%%%%%%%%%%%%%%%%%%%%%%%%%%%%
%
\begin{supertabular}{p{2,5cm}|ll}
%
\index{ala}
\textbf{\dots ala} && \textit{adjective numeral}: 0 \\ % no-dictionary
 && \\ % no-dictionary
%
\index{wan}
\textbf{\dots wan} && \textit{adjective numeral}: 1 \\ % no-dictionary
\textbf{wan} && \textit{noun}: unit, element, particle, part, piece \\ % no-dictionary
\textbf{wan (e \dots)} && \textit{verb transitive}: to unite, to make one \\ % no-dictionary
 && \\ % no-dictionary
%
\index{tu}
\textbf{\dots tu} && \textit{adjective numeral}: 2 \\ % no-dictionary
\textbf{tu} && \textit{noun}: duo, pair \\ % no-dictionary
\textbf{tu (e \dots)} && \textit{verb transitive}: to divide, to double, to separate, to cut in two \\ % no-dictionary
 && \\ % no-dictionary
%
\index{luka}
\textbf{\dots luka} && \textit{adjective numeral}: 5 \\ % no-dictionary
 && \\ % no-dictionary
%
\index{mute}
\textbf{\dots mute} && \textit{adjective numeral}: 20 (official Toki Pona book) \\ % no-dictionary
 && \\ % no-dictionary
%
\index{ale}
\textbf{\dots ale} && \textit{adjective numeral}: 100 (official Toki Pona book) \\ % no-dictionary
 && \\ % no-dictionary
%
\index{esun}
\textbf{\dots esun} && \textit{adjective}: marketable, for sale, salable, deductible \\  % no-dictionary
\textbf{esun} && \textit{noun}: market, shop, fair, bazaar, business, transaction \\ % no-dictionary
\textbf{esun (e \dots)} && \textit{verb transitive}: to buy, to sell, to barter, to swap \\ % no-dictionary
 && \\ % no-dictionary
%
\index{nanpa}
\textbf{nanpa \dots} && \textit{adjective numeral}: To build ordinal numbers. \\ % no-dictionary
\textbf{nanpa} && \textit{noun}: number, numeral \\ % no-dictionary
\textbf{nanpa (e \dots)} && \textit{verb transitive}: to count, to reckon,  to number \\ % no-dictionary
 && \\ % no-dictionary
%
\index{weka}
\textbf{\dots weka} && \textit{adjective}: absent, away, ignored \\ % no-dictionary
\textbf{weka} && \textit{noun}: absence \\ % no-dictionary
\textbf{weka (e \dots)} && \textit{verb transitive}: to remove, to eliminate, to throw away, to get rid of \\ % no-dictionary
 && \\ % no-dictionary 
%
\index{\#}
\textbf{\#} && \textit{unofficial}: Number sign  \\ % no-dictionary
%
\end{supertabular} \\
%
%
%
%%%%%%%%%%%%%%%%%%%%%%%%%%%%%%%%%%%%%%%%%%%%%%%%%%%%%%%%%%%%%%%%%%%%%%%%%%
\newpage
%
\subsection*{Use Numbers Sparingly!}
%
%%%%%%%%%%%%%%%%%%%%%%%%%%%%%%%%%%%%%%%%%%%%%%%%%%%%%%%%%%%%%%%%%%%%%%%%%%
%
The method that you're about to learn for making higher numbers should be avoided as much as possible. 
%
%%%%%%%%%%%%%%%%%%%%%%%%%%%%%%%%%%%%%%%%%%%%%%%%%%%%%%%%%%%%%%%%%%%%%%%%%%

\subsection*{Cardinal Numbers}
%%%%%%%%%%%%%%%%%%%%%%%%%%%%%%%%%%%%%%%%%%%%%%%%%%%%%%%%%%%%%%%%%%%%%%%%%%
%
There are only few number words in Toki Pona.
When we need to make higher numbers, we combine these numbers together. 

\begin{supertabular}{p{5,5cm}|ll}
wan && 1 \\ 
tu  && 2 \\ 
tu wan && 2 + 1 = 3 \\
tu tu && 2 + 2 = 4 \\
luka && 5 \\
luka wan && 5 + 1 = 6 \\
luka tu && 5 + 2 = 7 \\
luka tu wan && 5 + 2 + 1 = 8 \\
luka tu tu && 5 + 2 + 2 = 9 \\
luka luka && 5 + 5 = 10 \\
luka luka wan && 5 + 5 + 1 = 11 \\
luka luka tu && 5 + 5 + 2 = 12 \\
luka luka tu wan && 5 + 5 + 2 + 1 = 13 \\
luka luka tu tu && 5 + 5 + 2 + 2 = 14 \\
luka luka luka && 5 + 5 + 5 = 15 \\
mute wan && 20 + 1 = 21 (Is rarely used.) \\
ali tu && 100 + 2 = 102 (Is rarely used.) \\
\end{supertabular} 

These numbers are added onto nouns just like adjectives. 
When used together with other adjectives, numbers are inserted at the end.
Only pronouns can used after numbers.
You can insert unofficially a \# before numbers. 

\begin{supertabular}{p{5,5cm}|ll}
jan \# luka tu && 7 people \\
jan lili \# tu wan && 3 children \\
\end{supertabular} 

As you can see, it can get very confusing if you want to talk about numbers higher than 14 or so.
However, Toki Pona is simply not intended for such high numbers. 
It is a simple language. 
%
%%%%%%%%%%%%%%%%%%%%%%%%%%%%%%%%%%%%%%%%%%%%%%%%%%%%%%%%%%%%%%%%%%%%%%%%%%
% \newpage
%
\subsection*{Use \textit{mute}. Conserve the Numbers.}
%
\index{\textit{mute}}
%%%%%%%%%%%%%%%%%%%%%%%%%%%%%%%%%%%%%%%%%%%%%%%%%%%%%%%%%%%%%%%%%%%%%%%%%%
%
Okay, so it's a bad idea to use the numbers when you don't absolutely need them. 
So, instead, we use \textit{mute} for any number higher than two.

\begin{supertabular}{p{5,5cm}|ll}
jan mute li kama. && Many people came. \\
\end{supertabular} 

Of course, this is still pretty vague. 
\textit{mute} in the above sentence could mean 3 or it could mean 3 000. 
Fortunately, \textit{mute} is just an adjective, and so we can attach other adjectives after it. 
We have learned that you should not repeat a word. \textit{mute} and \textit{lili} are exceptions some people repeat it up to three times to represent higher numbers. 
This is not a good style. 
Better is to use \textit{mute kin}. 

\begin{supertabular}{p{5,5cm}|ll}
jan mute kin li kama! && Many, many, many people are coming! \\
\end{supertabular} 

More than likely, that sentence is saying that at least a thousand people are coming.  
Now suppose that you had more than two people but still not very many. 
Let's say that the number is around 4 or 5. Here's how you'd say that. 

\begin{supertabular}{p{5,5cm}|ll}
jan mute lili li kama. && A small amount (of) people are coming. \\
\end{supertabular} 
%
%%%%%%%%%%%%%%%%%%%%%%%%%%%%%%%%%%%%%%%%%%%%%%%%%%%%%%%%%%%%%%%%%%%%%%%%%%
%
\subsection*{Ordinal Numbers}
%
%%%%%%%%%%%%%%%%%%%%%%%%%%%%%%%%%%%%%%%%%%%%%%%%%%%%%%%%%%%%%%%%%%%%%%%%%%
%
If you understood how the cardinal numbers work, the ordinal numbers only require one more step. 
Like I said, if you understood the cardinal numbers, it's easy because you just stick \textit{nanpa} in between the noun and the number. 

\begin{supertabular}{p{5,5cm}|ll}
jan nanpa tu tu && 4th person \\
ni li jan lili ona nanpa tu. && This is her second child. \\
meli mi nanpa wan li ' nasa. && My first girlfriend was crazy. \\
\end{supertabular} 
%
%%%%%%%%%%%%%%%%%%%%%%%%%%%%%%%%%%%%%%%%%%%%%%%%%%%%%%%%%%%%%%%%%%%%%%%%%%
%
\subsection*{Other Uses of \textit{wan} and \textit{tu}}
%
\index{\textit{wan}!verb}
\index{\textit{tu}!verb}
%%%%%%%%%%%%%%%%%%%%%%%%%%%%%%%%%%%%%%%%%%%%%%%%%%%%%%%%%%%%%%%%%%%%%%%%%%
%
\textit{wan} can be used as a verb. 
It means 'to unite'. 

\begin{supertabular}{p{5,5cm}|ll}
mi en meli mi li ' wan. && My girlfriend and I got married. \\
jan pali pi ma ali o wan! && Proletarians of all countries, unite! \\

\end{supertabular} 

\textit{tu} used as a verb means 'to split' or 'to divide'. 

\begin{supertabular}{p{5,5cm}|ll}
o tu e palisa ni. && Split this stick.  \\
\end{supertabular} 

%%%%%%%%%%%%%%%%%%%%%%%%%%%%%%%%%%%%%%%%%%%%%%%%%%%%%%%%%%%%%%%%%%%%%%%%%%
%
\subsection*{The Meaning of Life}
%
\index{42}
%%%%%%%%%%%%%%%%%%%%%%%%%%%%%%%%%%%%%%%%%%%%%%%%%%%%%%%%%%%%%%%%%%%%%%%%%%

\begin{supertabular}{p{5,5cm}|ll}
ali li ' seme? &&  The Ultimate Question of Life, the Universe and Everything. \\
ni li ' mute mute tu.  && The answer is 42. \\
\end{supertabular} 
%
%
%
%%%%%%%%%%%%%%%%%%%%%%%%%%%%%%%%%%%%%%%%%%%%%%%%%%%%%%%%%%%%%%%%%%%%%%%%%%
\newpage
%
\subsection*{Miscellaneous}
%%%%%%%%%%%%%%%%%%%%%%%%%%%%%%%%%%%%%%%%%%%%%%%%%%%%%%%%%%%%%%%%%%%%%%%%%%
%
\index{\textit{weka}}
%
Today's word is \textit{weka}. 
As a verb, it just means 'to get rid of', 'to remove', etc. 

\begin{supertabular}{p{5,5cm}|ll}
o weka e len sina. && Remove your clothes. \\
o weka e jan lili, tan ni. && Remove the kid from here \\ 
ona li wile ala kute e ni. && He shouldn't hear this. \\ 
\end{supertabular} 

\index{\textit{weka}!adjective}
\index{\textit{weka}!adverb}
%
\textit{weka} is also used very often as an adjective and as an adverb. 

\begin{supertabular}{p{5,5cm}|ll}
mi ' weka. && I was away. \\
mi wile tawa weka. && I want to leave. \\
\end{supertabular} 

It can also be used to mean the equivalent of 'far' or 'distant'. 

\begin{supertabular}{p{5,5cm}|ll}
tomo mi li ' weka, tan ni. && My house is away from here. \\
ma Elopa li ' weka, tan ma Mewika. && Europe is away from the USA. \\
\end{supertabular} 

And add \textit{ala} to mean that it's somewhere closeby. 

\begin{supertabular}{p{5,5cm}|ll}
ma Mewika li ' weka ala, tan ma Kupa. && The USA is not away from Cuba. \\
\end{supertabular} 

%
%%%%%%%%%%%%%%%%%%%%%%%%%%%%%%%%%%%%%%%%%%%%%%%%%%%%%%%%%%%%%%%%%%%%%%%%%%
%
\subsubsection*{\textit{esun}}
%
\index{\textit{esun}}
%%%%%%%%%%%%%%%%%%%%%%%%%%%%%%%%%%%%%%%%%%%%%%%%%%%%%%%%%%%%%%%%%%%%%%%%%%

\begin{supertabular}{p{5,5cm}|ll}
mi nanpa e mani mi, lon esun suli. && I count my money at a supermarket. \\
\end{supertabular}

%
%
%
%%%%%%%%%%%%%%%%%%%%%%%%%%%%%%%%%%%%%%%%%%%%%%%%%%%%%%%%%%%%%%%%%%%%%%%%%%
\newpage
%
\subsection*{Practice (Answers: Page~\pageref{'numbers'})}
%%%%%%%%%%%%%%%%%%%%%%%%%%%%%%%%%%%%%%%%%%%%%%%%%%%%%%%%%%%%%%%%%%%%%%%%%%
%
Please write down your answers and check them afterwards. 

Try to translate these sentences. 
You can use the tool \textit{Toki Pona Parser} (\cite{www:rowa:02}) for spelling and grammar check. 

\begin{supertabular}{p{5,5cm}|ll}
I saw three birds.  &&   \\ % no-dictionary
Many people are coming. &&  \\   % no-dictionary
The first person is here. && \\   % no-dictionary
I own two cars.  &&  \\ % no-dictionary
Some (but not a lot) of people are coming. && \\  % no-dictionary  
Unite!  &&   \\ % no-dictionary
Is this a part? &&  \\ % no-dictionary
 && \\ % no-dictionary
mi weka e ijo tu ni. &&   \\ % no-dictionary
o tu.  &&  \\ % no-dictionary
mi lukin e soweli luka. && \\   % no-dictionary 
mi ' weka.  &&  \\ % no-dictionary
ona li sike ala sike? && \\ % no-dictionary
\end{supertabular}
%%%%%%%%%%%%%%%%%%%%%%%%%%%%%%%%%%%%%%%%%%%%%%%%%%%%%%%%%%%%%%%%%%%%%%%%%%
% eof

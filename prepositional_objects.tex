%%%%%%%%%%%%%%%%%%%%%%%%%%%%%%%%%%%%%%%%%%%%%%%%%%%%%%%%%%%%%%%%%%%%%%%%%%
\section{Prepositional Objects}
%%%%%%%%%%%%%%%%%%%%%%%%%%%%%%%%%%%%%%%%%%%%%%%%%%%%%%%%%%%%%%%%%%%%%%%%%%
\index{preposition}
\index{prepositional objects}
\index{\textit{lon}}
\index{\textit{kepeken}}
\index{\textit{tawa}}
\index{\textit{kama}}       
\index{\textit{kiwen}}
\index{\textit{kon}}
\index{\textit{pana}}
\index{\textit{poki}}
\index{\textit{toki}}
\index{exist}
\index{on}
\index{at}
\index{in}
\index{be!in}
\index{be!at}
\index{be!on}
\index{with}
\index{using}
\index{go}
\index{move}
\index{for}
\index{come}
\index{happen}
\index{cause}
\index{stone}
\index{rock}
\index{air}
\index{atmosphere}
\index{spirit}
\index{wind}
\index{square}
\index{block}
\index{stairs}
\index{give}
\index{send}
\index{release}
\index{emit}
\index{container}
\index{bowl}
\index{glass}
\index{cup}
\index{box}
\index{language}
\index{talk}
\index{speak}



%
%%%%%%%%%%%%%%%%%%%%%%%%%%%%%%%%%%%%%%%%%%%%%%%%%%%%%%%%%%%%%%%%%%%%%%%%%%
% \newpage
\subsection*{Prepositional Objekts and Prepositions}
\index{Object!prepositional}
\index{prepositional object}
\index{preposition}
%%%%%%%%%%%%%%%%%%%%%%%%%%%%%%%%%%%%%%%%%%%%%%%%%%%%%%%%%%%%%%%%%%%%%%%%%%
%
The third object class in \textit{toki pona} is the prepositional object. 
A prepositional object begins with a preposition. 
A preposition describes a relationship between other words in a sentence and stand in front of nouns or pronouns. 
It is closely connected to the verb. 
The preposition determines the case. 
The question of the prepositional object depends on the preposition used. 
In Toki Pona is a slot for prepositions only at the beginning of a prepositional object. 
It is recommended that you put a comma before a preposition.

In the prepositional object is the first slot after the preposition always a noun or pronoun slot.
After that, optional slots for adjectives, possessive pronouns and demonstrative pronouns are possible. 
In Toki Pona there is an optional prepositional object at the end of a sentence. 
Possible direct or indirect objects are always in front of a prepositional object. 
Like the other object types, a prepositional object is an optional part of a predicate phrase. 

%%%%%%%%%%%%%%%%%%%%%%%%%%%%%%%%%%%%%%%%%%%%%%%%%%%%%%%%%%%%%%%%%%%%%%%%%%
\index{preposition!\textit{kepeken}}
\index{\textit{kepeken}!preposition}

Using \textit{kepeken} as a preposition.

\begin{supertabular}{p{5,5cm}|ll}
mi moku, kepeken ilo moku. && I eat using a fork/spoon/ \\ && any type of eating utensil. \\
mi lukin, kepeken ilo suno. && I look using a flashlight.  \\
\end{supertabular} 

%
%%%%%%%%%%%%%%%%%%%%%%%%%%%%%%%%%%%%%%%%%%%%%%%%%%%%%%%%%%%%%%%%%%%%%%%%%%
\index{preposition!lon}
\index{\textit{lon}!preposition}
\index{\textit{lon}!\textit{wile}}
\index{\textit{wile}!\textit{lon}}
%%%%%%%%%%%%%%%%%%%%%%%%%%%%%%%%%%%%%%%%%%%%%%%%%%%%%%%%%%%%%%%%%%%%%%%%%%
%
\textit{lon} can be used as both a verb and a preposition. 
In this examples, \textit{lon} is used as a preposition.

\begin{supertabular}{p{5,5cm}|ll}
mi moku, lon tomo. && I eat in the house. \\
mi telo e mi, lon tomo telo. && I bathe myself in the restroom. \\
\end{supertabular} 

Because \textit{lon} can be used as either a preposition or a intrantitive verb, the meaning of the sentence can be a bit confusing when used with \textit{wile}. 

\begin{supertabular}{p{5,5cm}|ll}
mi wile lon tomo. && I want to be at home. / I want in a house. \\
\end{supertabular} 

\index{trick!\textit{e ni}}
\index{\textit{e}!\textit{ni}}
\index{\textit{ni}!\textit{e}}
The sentence has at least two possible translations. 
The first translation states that the speaker wishes he were at home. 
The second translation states that the speaker wants to do something in a house. 
We can use a comma to force \textit{lon} as a preposition.

\begin{supertabular}{p{5,5cm}|ll}
mi wile, lon tomo. && I want in a house. \\
\end{supertabular}

When you say, 'I want to be home.' you have to divide the sentence with a colon into two sentences.

\begin{supertabular}{p{5,5cm}|ll}
mi wile e ni: mi lon tomo. && I want this: I'm at home. \\
\end{supertabular} 

Toki Pona often uses this \textit{e ni:} trick. 
Before and after the colon has to be complete sentences. 
Toki Pona has no nested subordinate clauses.

\begin{supertabular}{p{5,5cm}|ll}
sina toki e ni, tawa mi: sina moku. && You told me that you are eating. \\
\end{supertabular} 

%%%%%%%%%%%%%%%%%%%%%%%%%%%%%%%%%%%%%%%%%%%%%%%%%%%%%%%%%%%%%%%%%%%%%%%%%%
\index{preposition!\textit{tawa}}
\index{verb!\textit{tawa}}
\index{\textit{tawa}!preposition}
\index{\textit{tawa}!verb}
\index{\textit{tawa}!to}
\index{to!\textit{tawa}}
\index{\textit{tawa}!for}
\index{verb!\textit{tawa}}
\index{\textit{tawa}!verb}
\index{\textit{tawa}!verb}
\index{verb!\textit{tawa}}
\index{\textit{tawa}!adjective}
\index{adjective!\textit{tawa}}
\index{car}
\index{boat}
\index{ship}
\index{airplane}
\index{helicopter}
%
In the last sentence \textit{tawa} is a preposition. 
Here are further examples.

\begin{supertabular}{p{5,5cm}|ll}
mi toki, tawa sina. && I talk to you. \\
ona li lawa e jan, tawa ma pona. && He led people to the good land. \\
ona li kama, tawa ma mi. && He's coming to my country. \\
\end{supertabular} 

In the following sentences the first \textit{tawa} is an intransitive verb.
The second \textit{tawa} is a preposition and initiates the prepositional object. 

\begin{supertabular}{p{5,5cm}|ll}
mi tawa, tawa tomo mi. && I'm going to my house. \\
ona mute li tawa, tawa utala. && They're going to the war. \\
sina wile tawa, tawa telo suli. && You want to go to the ocean. \\
ona li tawa, tawa sewi kiwen. && She's going up the rock. \\
\end{supertabular} 

In the following sentences the first \textit{tawa} is an transitive verb.
The second \textit{tawa} is a preposition.

\begin{supertabular}{p{5,5cm}|ll}
mi tawa e mi, tawa tomo mi. && I'm moving myself to my house. \\
mi tawa e kiwen, tawa sewi. && I'm moving the rock to the peak. \\
\end{supertabular} 

%
\index{\textit{li}!\textit{li pona tawa mi}}
\index{\textit{li}!\textit{li ike tawa mi}}
\index{like!I like}
\index{I like}
In Toki Pona, to say that you (don't) like something, we have pattern, and the pattern use \textit{tawa} as a preposition. 
This is done according to the pattern 'it is good to me' or 'it is bad to me'. 

\begin{supertabular}{p{5,5cm}|ll}
ni li ' pona, tawa mi. && That is good to me. / I like that. \\
ni li ' ike, tawa mi && That is bad to me. / I don't like that. \\
kili li ' pona, tawa mi. && I like fruit. \\
toki li ' pona, tawa mi. && I like talking. / I like languages. \\
utala li ' ike, tawa mi. && I don't like wars. \\
telo suli li ' ike, tawa mi. && I don't like the ocean. \\
\end{supertabular} 

%
\index{clauses}
Toki Pona does not use clauses. 
So for example, if you wanted to say 'I like watching the countryside,' it's best to split this into two sentences.

\begin{supertabular}{p{5,5cm}|ll}
mi lukin e ma. ni li ' pona, tawa mi. && I'm watching the countryside. This is good to me.\\
\end{supertabular} 

Of course, you could choose to say this same sentence using other techniques. 

\begin{supertabular}{p{5,5cm}|ll}
ma li pona lukin. && The countryside is good to look at. \\
\end{supertabular} 

%
The preposition \textit{tawa} can also mean 'for'.
 
\begin{supertabular}{p{5,5cm}|ll}
mi pona e tomo, tawa jan pakala. && I fixed the house for the disabled man. \\
\end{supertabular} 

% 
There are ambiguities since \textit{tawa} can also be used as an adjective. 
\textit{tawa} is used as an adjective to make the phrase we use for 'car', 'boat' or 'airplane'.

\begin{supertabular}{p{5,5cm}|ll}
tomo tawa && car (moving construction) \\
tomo tawa telo && boat, ship \\
tomo tawa kon && airplane, helicopter \\
\end{supertabular} 
%

Consider the following sentence.

\begin{supertabular}{p{5,5cm}|ll}
mi pana e tomo tawa sina. && ? \\   % no-dictionary
\end{supertabular} 

If \textit{tawa} is used as an adjective, then this sentence says 'I gave your car.' 
If it is used as a preposition, though, it could mean, 'I gave the house to you.' 
You can insert a comma before \textit{tawa} to mark it as a preposition. 
It is better to split the sentence. 

\begin{supertabular}{p{5,5cm}|ll}
mi jo e tomo tawa sina. mi pana e ni tawa sina. && I have your car. I give it to you. \\
ni li tomo. mi pana e ni tawa sina. && This is a house. I give it to you. \\
\end{supertabular} 

%
%%%%%%%%%%%%%%%%%%%%%%%%%%%%%%%%%%%%%%%%%%%%%%%%%%%%%%%%%%%%%%%%%%%%%%%%%%
\index{\textit{kama}}
\index{\textit{kama}!\textit{tawa}}
\index{\textit{tawa}!\textit{kama}}
%
In this sentence \textit{kama} is an intransitive verb and \textit{tawa} a preposition.

\begin{supertabular}{p{5,5cm}|ll}
ona li kama, tawa tomo mi. && He came to my house. \\
\end{supertabular} 


%
%%%%%%%%%%%%%%%%%%%%%%%%%%%%%%%%%%%%%%%%%%%%%%%%%%%%%%%%%%%%%%%%%%%%%%%%%%
\newpage
\subsection*{Practice (Answers: Page~\pageref{'prepositional_objects'})}
%%%%%%%%%%%%%%%%%%%%%%%%%%%%%%%%%%%%%%%%%%%%%%%%%%%%%%%%%%%%%%%%%%%%%%%%%%
%
Please write down your answers and check them afterwards. 

Try to translate these sentences. 
You can use the tool \textit{Toki Pona Parser} (\cite{www:rowa:02}) for spelling and grammar check. 

\begin{supertabular}{p{5,5cm}|ll}
I fixed the flashlight using a small tool.  &&  \\  % no-dictionary
I like Toki Pona.  &&  \\  % no-dictionary
We gave them food.  &&  \\ % no-dictionary
I want to go to his house using my car.  &&  \\  % no-dictionary
&& \\ % no-dictionary
sina wile kama, tawa tomo toki.  &&  \\ % no-dictionary
jan li toki, kepeken toki pona, lon tomo toki.  &&  \\ % no-dictionary
mi tawa, tawa tomo toki. ona li ' pona, tawa mi.  &&  \\ % no-dictionary
sina kama jo e jan pona, lon ni.  &&  \\ % no-dictionary
\end{supertabular} 

%
%%%%%%%%%%%%%%%%%%%%%%%%%%%%%%%%%%%%%%%%%%%%%%%%%%%%%%%%%%%%%%%%%%%%%%%%%%
% eof
